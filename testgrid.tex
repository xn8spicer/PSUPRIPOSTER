\documentclass[landscape,final,a0paper,fontscale=0.285]{baposter}

\usepackage{calc}
\usepackage{graphicx}
\usepackage{amsmath}
\usepackage{amssymb}
\usepackage{relsize}
\usepackage{multirow}
\usepackage{rotating}
\usepackage{bm}
\usepackage{url}
\usepackage{wrapfig}
\usepackage{array}
\usepackage{graphicx}
\usepackage{multicol}

%\usepackage{times}
%\usepackage{helvet}
%\usepackage{bookman}
\usepackage{palatino}

\newcommand{\captionfont}{\footnotesize}

\graphicspath{{images/}{../images/}}
\usetikzlibrary{calc}

\newcommand{\SET}[1]  {\ensuremath{\mathcal{#1}}}
\newcommand{\MAT}[1]  {\ensuremath{\boldsymbol{#1}}}
\newcommand{\VEC}[1]  {\ensuremath{\boldsymbol{#1}}}
\newcommand{\Video}{\SET{V}}
\newcommand{\video}{\VEC{f}}
\newcommand{\track}{x}
\newcommand{\Track}{\SET T}
\newcommand{\LMs}{\SET L}
\newcommand{\lm}{l}
\newcommand{\PosE}{\SET P}
\newcommand{\posE}{\VEC p}
\newcommand{\negE}{\VEC n}
\newcommand{\NegE}{\SET N}
\newcommand{\Occluded}{\SET O}
\newcommand{\occluded}{o}
\usepackage[font=small,labelfont=bf]{caption}

%%%%%%%%%%%%%%%%%%%%%%%%%%%%%%%%%%%%%%%%%%%%%%%%%%%%%%%%%%%%%%%%%%%%%%%%%%%%%%%%
%%%% Some math symbols used in the text
%%%%%%%%%%%%%%%%%%%%%%%%%%%%%%%%%%%%%%%%%%%%%%%%%%%%%%%%%%%%%%%%%%%%%%%%%%%%%%%%

%%%%%%%%%%%%%%%%%%%%%%%%%%%%%%%%%%%%%%%%%%%%%%%%%%%%%%%%%%%%%%%%%%%%%%%%%%%%%%%%
% Multicol Settings
%%%%%%%%%%%%%%%%%%%%%%%%%%%%%%%%%%%%%%%%%%%%%%%%%%%%%%%%%%%%%%%%%%%%%%%%%%%%%%%%
\setlength{\columnsep}{1.5em}
\setlength{\columnseprule}{0mm}

%%%%%%%%%%%%%%%%%%%%%%%%%%%%%%%%%%%%%%%%%%%%%%%%%%%%%%%%%%%%%%%%%%%%%%%%%%%%%%%%
% Save space in lists. Use this after the opening of the list
%%%%%%%%%%%%%%%%%%%%%%%%%%%%%%%%%%%%%%%%%%%%%%%%%%%%%%%%%%%%%%%%%%%%%%%%%%%%%%%%
\newcommand{\compresslist}{%
\setlength{\itemsep}{1pt}%
\setlength{\parskip}{0pt}%
\setlength{\parsep}{0pt}%
}

%%%%%%%%%%%%%%%%%%%%%%%%%%%%%%%%%%%%%%%%%%%%%%%%%%%%%%%%%%%%%%%%%%%%%%%%%%%%%%
%%% Begin of Document
%%%%%%%%%%%%%%%%%%%%%%%%%%%%%%%%%%%%%%%%%%%%%%%%%%%%%%%%%%%%%%%%%%%%%%%%%%%%%%

\begin{document}

%%%%%%%%%%%%%%%%%%%%%%%%%%%%%%%%%%%%%%%%%%%%%%%%%%%%%%%%%%%%%%%%%%%%%%%%%%%%%%
%%% Here starts the poster
%%%---------------------------------------------------------------------------
%%% Format it to your taste with the options
%%%%%%%%%%%%%%%%%%%%%%%%%%%%%%%%%%%%%%%%%%%%%%%%%%%%%%%%%%%%%%%%%%%%%%%%%%%%%%
% Define some colors

%\definecolor{lightblue}{cmyk}{0.83,0.24,0,0.12}
\definecolor{lightblue}{rgb}{0.145,0.6666,1}


\hyphenation{resolution occlusions}
%%
\begin{poster}%
  % Poster Options
  {
  % Show grid to help with alignment
  grid=false,
  % Column spacing
  colspacing=1em,
  % Color style
  bgColorOne=white,
  bgColorTwo=white,
  borderColor=lightblue,
  headerColorOne=black,
  headerColorTwo=lightblue,
  headerFontColor=white,
  boxColorOne=white,
  boxColorTwo=lightblue,
  % Format of textbox
  textborder=roundedleft,
  % Format of text header
  eyecatcher=true,
  headerborder=closed,
  headerheight=0.1\textheight,
%  textfont=\sc, An example of changing the text font
  headershape=roundedright,
  headershade=shadelr,
  headerfont=\Large\bf\textsc, %Sans Serif
  textfont={\setlength{\parindent}{1.5em}},
  boxshade=plain,
%  background=shade-tb,
  background=plain,
  linewidth=2pt
  }
  % University logo in Eyecatcher location
  {\includegraphics[height=3em]{images/psulogo.png}} 
  % Title
  {\bf\textsc{GraphTrack: Fast and Globally Optimal Tracking in Videos}\vspace{0.5em}}
  % Authors
  {\textsc{\{ Brian.Amberg and Thomas.Vetter \}@unibas.ch}}
  {% The makebox allows the title to flow into the logo, this is a hack because of the L shaped logo.
   % Eye Catcher in Logo location
%    \includegraphics[height=4.0em]{images/pri-world.jpeg}
  }

%%%%%%%%%%%%%%%%%%%%%%%%%%%%%%%%%%%%%%%%%%%%%%%%%%%%%%%%%%%%%%%%%%%%%%%%%%%%%%
%%% Now define the boxes that make up the poster
%%%---------------------------------------------------------------------------
%%% Each box has a name and can be placed absolutely or relatively.
%%% The only inconvenience is that you can only specify a relative position 
%%% towards an already declared box. So if you have a box attached to the 
%%% bottom, one to the top and a third one which should be in between, you 
%%% have to specify the top and bottom boxes before you specify the middle 
%%% box.
%%%%%%%%%%%%%%%%%%%%%%%%%%%%%%%%%%%%%%%%%%%%%%%%%%%%%%%%%%%%%%%%%%%%%%%%%%%%%%
    %
    % A coloured circle useful as a bullet with an adjustably strong filling
    \newcommand{\colouredcircle}{%
      \tikz{\useasboundingbox (-0.2em,-0.32em) rectangle(0.2em,0.32em); \draw[draw=black,fill=lightblue,line width=0.03em] (0,0) circle(0.18em);}}

%%%%%%%%%%%%%%%%%%%%%%%%%%%%%%%%%%%%%%%%%%%%%%%%%%%%%%%%%%%%%%%%%%%%%%%%%%%%%%
%
%  ALL COLUMNS OF TOP ROW
%
%%%%%%%%%%%%%%%%%%%%%%%%%%%%%%%%%%%%%%%%%%%%%%%%%%%%%%%%%%%%%%%%%%%%%%%%%%%%%%
  \headerbox{Position00}{name=p00,column=0,row=0}{
%%%%%%%%%%%%%%%%%%%%%%%%%%%%%%%%%%%%%%%%%%%%%%%%%%%%%%%%%%%%%%%%%%%%%%%%%%%%%%
   Numbering is 0-3 for a four columns.  So top, first column c=0
    }
%%%%%%%%%%%%%%%%%%%%%%%%%%%%%%%%%%%%%%%%%%%%%%%%%%%%%%%%%%%%%%%%%%%%%%%%%%%%%%
  \headerbox{Position10}{name=p10,column=1,row=0}{
%%%%%%%%%%%%%%%%%%%%%%%%%%%%%%%%%%%%%%%%%%%%%%%%%%%%%%%%%%%%%%%%%%%%%%%%%%%%%%
   Top, second column is c=1
     \noindent{
      \centering
      \includegraphics[width=0.95\linewidth]{images/graph_occluded.pdf}\\}
    }

%%%%%%%%%%%%%%%%%%%%%%%%%%%%%%%%%%%%%%%%%%%%%%%%%%%%%%%%%%%%%%%%%%%%%%%%%%%%%%
  \headerbox{Position20}{name=p20,column=2,row=0}{
%%%%%%%%%%%%%%%%%%%%%%%%%%%%%%%%%%%%%%%%%%%%%%%%%%%%%%%%%%%%%%%%%%%%%%%%%%%%%%
   Top third column, c=2

    }
%%%%%%%%%%%%%%%%%%%%%%%%%%%%%%%%%%%%%%%%%%%%%%%%%%%%%%%%%%%%%%%%%%%%%%%%%%%%%%
  \headerbox{Position30}{name=p30,column=3,row=0}{
%%%%%%%%%%%%%%%%%%%%%%%%%%%%%%%%%%%%%%%%%%%%%%%%%%%%%%%%%%%%%%%%%%%%%%%%%%%%%%
   Top fourth column, c=3.  Can have 6, but default is 4
    }

%%%%%%%%%%%%%%%%%%%%%%%%%%%%%%%%%%%%%%%%%%%%%%%%%%%%%%%%%%%%%%%%%%%%%%%%%%%%%%
%
% ALL COLUMNS OF BOTTOM ROW, RELATIVE TO BOTTOM BY SAYING above=bottom
%
%%%%%%%%%%%%%%%%%%%%%%%%%%%%%%%%%%%%%%%%%%%%%%%%%%%%%%%%%%%%%%%%%%%%%%%%%%%%%%
  \headerbox{Position12}{name=p12,column=1,above=bottom} {
%%%%%%%%%%%%%%%%%%%%%%%%%%%%%%%%%%%%%%%%%%%%%%%%%%%%%%%%%%%%%%%%%%%%%%%%%%%%%%
   Bottom row, second column
    }

%%%%%%%%%%%%%%%%%%%%%%%%%%%%%%%%%%%%%%%%%%%%%%%%%%%%%%%%%%%%%%%%%%%%%%%%%%%%%%
%
% ALL COLUMNS OF MIDDLE ROW, RELATIVE TO ITEMS ABOVE AND BELOW.   BE CAREFUL WITH SECOND AND THIRD ROWS
% You cannot refer to something relative until it has been defined!!!!
%
%%%%%%%%%%%%%%%%%%%%%%%%%%%%%%%%%%%%%%%%%%%%%%%%%%%%%%%%%%%%%%%%%%%%%%%%%%%%%%
     
    
%%%%%%%%%%%%%%%%%%%%%%%%%%%%%%%%%%%%%%%%%%%%%%%%%%%%%%%%%%%%%%%%%%%%%%%%%%%%%%
  \headerbox{Position11}{name=p11,column=1,span=2,row=0, below=p10}{
%%%%%%%%%%%%%%%%%%%%%%%%%%%%%%%%%%%%%%%%%%%%%%%%%%%%%%%%%%%%%%%%%%%%%%%%%%%%%%
   Second column, second(middle) row
   
% latex table generated in R 2.15.2 by xtable 1.7-0 package
% Fri Dec 14 16:14:46 2012
\begin{minipage}{\linewidth}
\centering
\begin{tabular}{lrrrrr}
  \hline
 & Df & Sum Sq & Mean Sq & F value & Pr($>$F) \\ 
  \hline
sex & 1 & 75.37 & 75.37 & 0.38 & 0.5417 \\ 
  ethnicty & 3 & 2572.15 & 857.38 & 4.27 & 0.0072 \\ 
  grade & 1 & 36.31 & 36.31 & 0.18 & 0.6717 \\ 
  disadvg & 1 & 59.30 & 59.30 & 0.30 & 0.5882 \\ 
  Residuals & 93 & 18682.87 & 200.89 &  &  \\ 
   \hline
\end{tabular}
\end{minipage}
}
 %%%%%%%%%%%%%%%%%%%%%%%%%%%%%%%%%%%%%%%%%%%%%%%%%%%%%%%%%%%%%%%%%%%%%%%%%%%%%%
  \headerbox{Position15}{name=p15,column=1,span=2,row=0, below=p11, above=p12}{
%%%%%%%%%%%%%%%%%%%%%%%%%%%%%%%%%%%%%%%%%%%%%%%%%%%%%%%%%%%%%%%%%%%%%%%%%%%%%%
   Second column, third row
   \begin{center}
   \includegraphics[height=4em]{images/priqr.png}
   \end{center}
   
    }
    


\end{poster}

\end{document}
